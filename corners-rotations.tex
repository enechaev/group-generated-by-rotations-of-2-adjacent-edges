\documentclass[utf8,a4paper,draft]{article}
\usepackage[utf8]{inputenc}
\usepackage[russian]{babel}
\usepackage[]{amsmath,amssymb,textcomp,amsthm}
\title{О состояниях поворотов угловых кубиков}
\author{}
\date{}

\newtheorem*{lemma1}{Лемма о поворотах двух смежных кубиков}
\newtheorem*{lemma2}{Лемма о количестве разбиений}
\newtheorem*{lemma3}{О группе}
\usepackage{amsmath}
\begin{document}
\maketitle
\begin{lemma1}
У любой перестановки можно поменять углы поворотов двух смежных кубиков.
\end{lemma1}
\begin{proof}
Существует такая комбинация поворотов:
\begin{multline*}
$$
M = ULU^{-1}LU^{-1}L^{-1}U^2L^{-1}U^{-1}LU^{-1}LULU^{-1}       \\
    L^{-1}UL^{-1}U^{-1}L^2ULU^{-1}L^2ULU^{-1}LU^2LU^{-1}L^{-1} \\
    U^{-1}L^2ULUL^{-1}U^2L^2U^2LU^{-1}LUL^2U^{-1}L^2
$$
\end{multline*}
    Нетрудно убедиться, что эта комбинация оставляет на месте все кубики, но
два кубика, общих для обеих граней, поворачивает: 1 на 240\textdegree, другой
на 120\textdegree.

    Если есть движение $N$, которое двигает 2 смежных кубика на смежное для обеих граней ребро,
то их можно повернуть комбинацией $NMN^{-1}$ ($NM^2N^{-1}$).
    А повороты $N$~--- это повороты одной из граней 1, 2 и 3 раза.
\end{proof}
С помощью вот таких поворотов можно к произвольной комбинации поворотов
прибавить 3 к любым из 6 кубиков. То есть можно получить столько же комбинаций,
сколько разбиений чисел 0,3,6,9,12 на 6 слагаемых из множества $\{0,1,2\}$ с
учетом порядка.
\begin{lemma2}
    Разбиений чисел 0,3,6,9,12 на 6 слагаемых из множества $\{0,1,2\}$~--- 243.
\end{lemma2}
\begin{proof}
    Разбиений по суммам(0~-- поворот на 0\textdegree; 1~-- на 120\textdegree; 2~-- на 240\textdegree):
    \begin{description}
    \item[0.] 1 разбиение: $[0,0,0,0,0,0]$.
    \item[3.] 2 разбиения: $[2,1,0,0,0,0]$, $[1,1,1,0,0,0]$. Всего вариантов
        $A^2_6+C^3_6=50$.
    \item[6.] 4 разбиения: $[1,1,2,2,0,0]$, $[2,2,2,0,0,0]$, $[1,1,1,1,2,0]$,
        $[1,1,1,1,1,1]$. Всего вариантов $1+A^2_6+C^3_6+C^2_6C^2_4=141$.
    \item[9.] 2 разбиения: $[1,1,1,2,2,2]$, $[2,2,2,2,1,0]$. Всего вариантов
        $A^2_6+C^3_6=50$.
    \item[12.]1 разбиение: $[2,2,2,2,2,2]$.
    \end{description}
    Всего вариантов 243.
\end{proof}
\begin{lemma3}
Группа комбинаций поворотов маленьких кубиков~--- $\mathbb{Z}_3^5$.
\end{lemma3}
\begin{proof}
    Каждый элемент в ней имеет порядок 3 и размер группы $3^5$, поэтому она изоморфна $\mathbb{Z}_3^5$.
\end{proof}
\end{document}
