\documentclass[utf8,russian]{beamer}
\usepackage[utf8]{inputenc}
\usepackage[russian]{babel}

\mode<presentation> {
\usetheme{Madrid}
}

\usepackage{graphicx}
\usepackage{booktabs}

\usepackage{tikz}
\usepackage{rubikcube,rubikrotation,rubikpatterns}
\usepackage[]{amsmath,amssymb,textcomp,amsthm}

\newtheorem{ru_theo}{Лемма.}
\renewenvironment{theorem}{\begin{ru_theo}}{\end{ru_theo}}

\title[Подгруппа кубика Рубика]{Группа, порожденная поворотами двух смежных граней кубика Рубика}
\author{Егор Нечаев}\institute[СУНЦ МГУ]{
Специализированный учебно-научный центр (факультет) – школа-интернат имени А.Н.Колмогорова Московского государственного университета имени М.В.Ломоносова \\
\medskip
\textit{nechaev.e.01@gmail.com}
}
\date{Москва, 24.02.2019}

\begin{document}
\begin{frame}
\titlepage
\RubikCubeSolved
%\RubikRotation{y3,z3}
\RubikRotation{random,50}
\begin{figure}
\ShowCube{2cm}{0.5}{\DrawRubikCubeLU}
\end{figure}
\end{frame}
\begin{frame}
\frametitle{Введение}
Цель работы: исследовать подгруппу группы кубика Рубика, порожденную поворотами двух смежных граней.

Задачи: рассмотреть группы движения отдельных элементов
\begin{figure}[ht]
\centering
\begin{tikzpicture}[->,auto,scale=1,semithick]
\node[text width=2cm,align=center] (a) at (1.5,3) {Движения граней};
\node[text width=2cm,align=center] (b) at (-3,1) {Движения середин};
\node[text width=2cm,align=center] (c) at (0,1) {Повороты середин};
\node[text width=2cm,align=center] (d) at (3,1) {Движения углов};
\node[text width=2cm,align=center] (e) at (6,1) {Повороты углов};
\path   (a)  edge node      {} (b)
        (a)  edge node      {} (c)
        (a)  edge node      {} (d)
        (a)  edge node      {} (e)
;
\end{tikzpicture}
\caption{Отдельные элементы\label{states_graph}}
\end{figure}
\end{frame}

%\begin{frame}
%	\frametitle{Обозначения}
%	\RubikCubeSolved
%	\RubikRotation{y3,z3}
%	\begin{figure}
%		\ShowCube{10cm}{1}{
%			\DrawRubikCubeLU
%		}
%	\end{figure}
%\end{frame}

\section{centers}
\subsection{centers.perms}

\begin{frame}
\begin{theorem}
	Группа перестановок средних кубиков изоморфна $S_7$.
\end{theorem}
\frametitle{Группа перестановок средних кубиков}
\begin{proof}
G действует на 7 элементах, поэтому $G\subseteq S_7$. Кроме того существуют движения $UL=(1,2,3,4,5,6,7)$ и $L^{-1}U^2LU^{-1}L^{-1}U^{-1}LU^{-1}=(1,2)$.
$(1,2,\ldots,n)^{-i}(1,2)(1,2,\ldots,n)^i=(i+1,i+2)$~--- порождающие элементы $S_7$, поэтому $S_7\subseteq G$.
\end{proof}
\vspace*{-5mm}
\begin{columns}[c]
\column{.45\textwidth}
\begin{figure}
\RubikCubeSolved
\RubikRotation{U,L}
\RubikRotation{y3,z3}
\ShowCube{2cm}{0.3}{\DrawRubikCubeF}
\caption{$UL$}
\end{figure}
\column{.45\textwidth}
\begin{figure}
\RubikCubeSolved
\RubikRotation{Lp,U2,L,Up,Lp,Up,L,Up}
\RubikRotation{y3,z3}
\ShowCube{2cm}{0.3}{\DrawRubikCubeF}
\caption{$L^{-1}U^2LU^{-1}L^{-1}U^{-1}LU^{-1}$}
\end{figure}
\end{columns}
\end{frame}

%------------------------------------------------

\begin{frame}
\frametitle{Группа поворотов средних кубиков}
\begin{theorem}
Группа поворотов средних кубиков тривиальна.
\end{theorem}
\begin{proof}
	\RubikCubeSolved
	\RubikRotation{y3,z3}
	\begin{figure}
		\ShowCube{2cm}{0.5}{
			\DrawRubikCubeLU
			\draw[->,ultra thick,color=green] (0,1.25) -- (0,1.75);
			\draw[->,ultra thick,color=green] (1.75,0) -- (1.25,0);
			\draw[->,ultra thick,color=green] (3,1.75) -- (3,1.25);
			\draw[->,ultra thick,color=green] (1.25,3) -- (1.75,3);
			\draw[->,ultra thick,color=green] (-0.7,0.7) -- (-0.5,0.5);
			\draw[->,ultra thick,color=green] (-1,2.75) -- (-1,2.25);
			\draw[->,ultra thick,color=green] (-0.4,3.4) -- (-0.7,3.7);
		}
	\end{figure}
	Чтобы у среднего кубика поменялась ориентация, нужно, чтобы поменялось направление стрелочки. При поворотах граней стрелочки всегда совмещаются.
\end{proof}
\end{frame}

%------------------------------------------------

\begin{frame}
\frametitle{Группа поворотов угловых кубиков}
\begin{theorem}
Группа поворотов угловых кубиков изоморфна $\mathbb{Z}_3^5$.
\end{theorem}
\begin{multline*}
$$
M = ULU^{-1}LU^{-1}L^{-1}U^2L^{-1}U^{-1}LU^{-1}LULU^{-1}       \\
L^{-1}UL^{-1}U^{-1}L^2ULU^{-1}L^2ULU^{-1}LU^2LU^{-1}L^{-1} \\
U^{-1}L^2ULUL^{-1}U^2L^2U^2LU^{-1}LUL^2U^{-1}L^2
$$
\end{multline*}
\begin{figure}
	\RubikCubeSolved
	\RubikRotation{U,L,Up,L,Up,Lp,U2,Lp,Up,L,Up,L,U,L,Up,
		Lp,U,Lp,Up,L2,U,L,Up,L2,U,L,Up,L,U2,L,Up,Lp,
		Up,L2,U,L,U,Lp,U2,L2,U2,L,Up,L,U,L2,Up,L2}
	\RubikRotation{y3,z3}
	\ShowCube{2cm}{0.5}{\DrawRubikCubeLU}
	\caption{Движение $M$}
\end{figure}
\end{frame}

%------------------------------------------------
\section{corners.perms}
%------------------------------------------------

\begin{frame}
\frametitle{Группа перестановок угловых кубиков}
\RubikCubeSolved
\RubikRotation{y3,z3}
\begin{figure}
	\ShowCube{10cm}{1}{
		\DrawRubikCubeLU
		\draw[->,ultra thick,color=green] (0.5,2.5) -- (0.5,0.5);
		\draw[->,ultra thick,color=green] (2.5,2.5) -- (2.5,0.5);
		\draw[->,ultra thick,color=green] (-0.85,3.4) -- (-0.85,1.15);
	}
\end{figure}
\end{frame}

\begin{frame}
%\frametitle{Группа перестановок угловых кубиков}
Разделим все кубики на 20 групп.
\vspace*{-5mm}
\begin{columns}[c]
\column{.5\textwidth}
\begin{figure}[c]
	\centering
	\scalebox{.45}{
	\begin{picture}(300,380)
	\put(5,370){\circle*{5}} \put(25,370){\circle*{5}} \put(45,370){\circle*{5}}
	\put(5,350){\circle*{5}} \put(25,350){\circle*{5}} \put(45,350){\circle*{5}}
	\put(5,350){\line(0,0){20}} \put(25,350){\line(0,0){20}} \put(45,350){\line(0,0){20}}
	\put(270,357){(0)}
	
	\put(5,320){\circle*{5}} \put(25,320){\circle*{5}} \put(45,320){\circle*{5}}
	\put(75,320){\circle*{5}} \put(95,320){\circle*{5}} \put(115,320){\circle*{5}}
	\put(5,300){\line(0,0){20}} \put(25,300){\line(1,0){20}} \put(25,320){\line(1,0){20}}
	\put(5,300){\circle*{5}} \put(25,300){\circle*{5}} \put(45,300){\circle*{5}}
	\put(75,300){\circle*{5}} \put(95,300){\circle*{5}} \put(115,300){\circle*{5}}
	\put(115,300){\line(0,0){20}} \put(95,300){\line(-1,0){20}} \put(95,320){\line(-1,0){20}}
	\put(270,317){(1 a, b)}
	
	\put(5,270){\circle*{5}} \put(25,270){\circle*{5}} \put(45,270){\circle*{5}}
	\put(75,270){\circle*{5}} \put(95,270){\circle*{5}} \put(115,270){\circle*{5}}
	\put(5,250){\line(2,1){41}} \put(25,250){\line(-1,1){20}} \put(45,250){\line(-1,1){20}}
	\put(5,250){\circle*{5}} \put(25,250){\circle*{5}} \put(45,250){\circle*{5}}
	\put(75,250){\circle*{5}} \put(95,250){\circle*{5}} \put(115,250){\circle*{5}}
	\put(115,250){\line(-2,1){41}} \put(95,250){\line(1,1){20}} \put(75,250){\line(1,1){20}}
	\put(270,257){(2 a, b)}
	
	\put(5,220){\circle*{5}} \put(25,220){\circle*{5}} \put(45,220){\circle*{5}}
	\put(5,200){\circle*{5}} \put(25,200){\circle*{5}} \put(45,200){\circle*{5}}
	\put(5,200){\line(2,1){41}} \put(45,200){\line(-2,1){41}} \put(25,200){\line(0,1){20}}
	\put(270,207){(3)}
	
	\put(5,170){\circle*{5}} \put(25,170){\circle*{5}} \put(45,170){\circle*{5}}
	\put(5,150){\circle*{5}} \put(25,150){\circle*{5}} \put(45,150){\circle*{5}}
	\put(5,150){\line(1,0){20}} \put(45,150){\line(-1,1){20}} \qbezier(5,170)(25,190)(45,170)
	\put(75,170){\circle*{5}} \put(95,170){\circle*{5}} \put(115,170){\circle*{5}}
	\put(75,150){\circle*{5}} \put(95,150){\circle*{5}} \put(115,150){\circle*{5}}
	\put(75,150){\line(1,1){20}} \put(95,150){\line(1,0){20}} \qbezier(75,170)(95,190)(115,170)
	\put(145,170){\circle*{5}} \put(165,170){\circle*{5}} \put(185,170){\circle*{5}}
	\put(145,150){\circle*{5}} \put(165,150){\circle*{5}} \put(185,150){\circle*{5}}
	\put(165,150){\line(-1,1){20}} \put(165,170){\line(1,0){20}} \qbezier(145,150)(165,130)(185,150)
	\put(215,170){\circle*{5}} \put(235,170){\circle*{5}} \put(255,170){\circle*{5}}
	\put(215,150){\circle*{5}} \put(235,150){\circle*{5}} \put(255,150){\circle*{5}}
	\put(215,170){\line(1,0){20}} \put(235,150){\line(1,1){20}} \qbezier(215,150)(235,130)(255,150)
	\put(270,157){(4 a, b, c, d)}
	
	\put(5,120){\circle*{5}} \put(25,120){\circle*{5}} \put(45,120){\circle*{5}}
	\put(5,100){\circle*{5}} \put(25,100){\circle*{5}} \put(45,100){\circle*{5}}
	\put(5,120){\line(2,-1){41}} \put(5,100){\line(1,0){20}} \put(25,120){\line(1,0){20}}
	\put(75,120){\circle*{5}} \put(95,120){\circle*{5}} \put(115,120){\circle*{5}}
	\put(75,100){\circle*{5}} \put(95,100){\circle*{5}} \put(115,100){\circle*{5}}
	\put(75,100){\line(2,1){41}} \put(75,120){\line(1,0){20}} \put(95,100){\line(1,0){20}}
	\put(270,107){(5 a, b)}
	
	\put(5,70){\circle*{5}} \put(25,70){\circle*{5}} \put(45,70){\circle*{5}}
	\put(5,50){\circle*{5}} \put(25,50){\circle*{5}} \put(45,50){\circle*{5}}
	\qbezier(5,70)(25,90)(45,70) \qbezier(5,50)(25,30)(45,50) \put(25,50){\line(0,1){20}}
	\put(270,57){(6)}
	
	\put(5,20){\circle*{5}} \put(25,20){\circle*{5}} \put(45,20){\circle*{5}}
	\put(5,0){\circle*{5}} \put(25,0){\circle*{5}} \put(45,0){\circle*{5}}
	\put(5,0){\line(0,1){20}} \put(25,0){\line(1,1){20}} \put(45,0){\line(-1,1){20}}
	\put(75,20){\circle*{5}} \put(95,20){\circle*{5}} \put(115,20){\circle*{5}}
	\put(75,0){\circle*{5}} \put(95,0){\circle*{5}} \put(115,0){\circle*{5}}
	\put(115,0){\line(0,1){20}} \put(75,0){\line(1,1){20}} \put(95,0){\line(-1,1){20}}
	\put(270,17){(7 a, b)}
	\end{picture}
	}
	\centering
	\caption{\small{Взаимные расположения групп угловых кубиков\label{possibleperms}}}
\end{figure}
\column{.54\textwidth}
\begin{figure}[c]
	\centering
	\scalebox{.66}{
	\begin{tikzpicture}[->,auto,scale=1,semithick]
	\node (7b) at (5,-2) {7b};
	\node (7a) at (1,-2) {7a};
	\node (6) at (3,-4) {6};
	\node (5b) at (5,-3) {5b};
	\node (5a) at (1,-3) {5a};
	\node (2b) at (1,-1) {2b};
	\node (2a) at (5,-1) {2a};
	\node (3) at (3,3) {3};
	\node (4a) at (0,1) {4a};
	\node (4d) at (2,1) {4d};
	\node (4c) at (4,1) {4c};
	\node (4b) at (6,1) {4b};
	\node (1a) at (1,4) {1a};
	\node (1b) at (5,4) {1b};
	\node (0) at (3,5) {0};
	
	\path   (0)  edge [bend left=10] node[xshift=-10pt]      {\scriptsize{$L$,$L^{-1}$}} (1a)
	edge [bend left=10] node                    {\scriptsize{$U$,$U^{-1}$}} (1b)
	(1a) edge                node[above,xshift=5pt]  {\scriptsize{$U$}}          (4d)
	edge                node[above]             {\scriptsize{$U^{-1}$}}     (4a)
	edge [bend left=20] node                    {\scriptsize{$L$,$L^{-1}$}} (0)
	(1b) edge                node[above]             {\scriptsize{$L^{-1}$}}     (4c)
	edge                node                    {\scriptsize{$L$}}          (4b)
	edge [bend left=20] node[xshift=10pt]       {\scriptsize{$U$,$U^{-1}$}} (0)
	(4a) edge                node[below]             {\scriptsize{$U^{-1}$}}     (3)
	edge                node[below]             {\scriptsize{$L$}}          (2b)
	edge                node[below]             {\scriptsize{$L^{-1}$}}     (4d)
	(4b) edge                node[below]             {\scriptsize{$L$}}          (3)
	edge                node                    {\scriptsize{$U^{-1}$}}     (2a)
	(4c) edge                node[below,xshift=-6pt] {\scriptsize{$L^{-1}$}}     (3)
	edge                node[above]             {\scriptsize{$U$}}          (2b)
	edge                node[below]             {\scriptsize{$U^{-1}$}}     (4b)
	(4d) edge                node[below,xshift=4pt]  {\scriptsize{$U$}}          (3)
	edge                node                    {\scriptsize{$L^{-1}$}}     (2a)
	(2b) edge [bend left=10] node                    {\scriptsize{$U$}}          (2a)
	(2a) edge [bend left=10] node                    {\scriptsize{$U^{-1}$}}     (2b)
	
	(7a) edge [loop left]    node                    {\scriptsize{$L$,$L^{-1}$}} (7a)
	edge                node                    {\scriptsize{$U^{-1}$}}     (5a)
	edge                node                    {\scriptsize{$U$}}          (5b)
	(5a) edge                node[below]             {\scriptsize{$U^{-1}$,$L^{-1}$}} (6)
	(7b) edge [loop right]   node                    {\scriptsize{$U$,$U^{-1}$}} (7b)
	edge                node                    {\scriptsize{$L$}}          (5b)
	edge                node                    {\scriptsize{$L^{-1}$}}     (5a)
	(5b) edge                node                    {\scriptsize{$U$,$L$}}      (6)
	;
	\end{tikzpicture}
	}
	\caption{Граф переходов между состояниями\label{states_graph}}
\end{figure}
\end{columns}
\end{frame}

%------------------------------------------------

\begin{frame}
\begin{theorem}
Для разных перестановок $a_1$, $a_2$ из подкласса $0+$ и из одной перестановки любого класса $b$ $a_1b$ и $a_2b$~--- разные перестановки того же подкласса, что и $b$.
\end{theorem}
\begin{theorem}
В подклассе 0+ возможно 6 разных состояний.
\end{theorem}
\begin{proof}
Если для состояния $M$ возможно состояние $F$, в котором только у одной из стрелочек изменено направление, то $MF^{-1}$~--- 2-цикл. Все 2-циклы~--- либо неразрешенные состояния, либо переводятся в неразрешенные.

2-цикл $C$, меняющий местами два соседних горизонтальных кубика, принадлежит к классам 7a,b.
\begin{columns}
\column{.45\textwidth}
\begin{figure}[c]
	\begin{tikzpicture}
	\node[style={shape=circle,fill=black,scale=0.2}] (A2) at (0,0) {};
	\node[style={shape=circle,fill=black,scale=0.2}] (A1) at (0,0.5) {};
	\node[style={shape=circle,fill=black,scale=0.2}] (B2) at (0.5,0) {};
	\node[style={shape=circle,fill=black,scale=0.2}] (B1) at (0.5,0.5) {};
	\node[style={shape=circle,fill=black,scale=0.2}] (C2) at (1,0) {};
	\node[style={shape=circle,fill=black,scale=0.2}] (C1) at (1,0.5) {};
	\draw[->] (A1) -- (A2);
	\draw[->] (B1) -- (B2);
	\draw[->] (C2) -- (C1);
	\path[->] (1.3,0.25) edge node[above] {\tiny{$L$}} (1.7,0.25);
	\node[style={shape=circle,fill=black,scale=0.2}] (A2) at (2,0) {};
	\node[style={shape=circle,fill=black,scale=0.2}] (A1) at (2,0.5) {};
	\node[style={shape=circle,fill=black,scale=0.2}] (B2) at (2.5,0) {};
	\node[style={shape=circle,fill=black,scale=0.2}] (B1) at (2.5,0.5) {};
	\node[style={shape=circle,fill=black,scale=0.2}] (C2) at (3,0) {};
	\node[style={shape=circle,fill=black,scale=0.2}] (C1) at (3,0.5) {};
	\draw[->] (A1) -- (A2);
	\draw[->] (C1) -- (B1);
	\draw[->] (B2) -- (C2);
	\node (e) at (3.5,0.25) {~$=LC$};
	\end{tikzpicture}
\end{figure}
\column{.45\textwidth}
\begin{figure}[c]
	\begin{tikzpicture}
	\node[style={shape=circle,fill=black,scale=0.2}] (A2) at (0,0) {};
	\node[style={shape=circle,fill=black,scale=0.2}] (A1) at (0,0.5) {};
	\node[style={shape=circle,fill=black,scale=0.2}] (B2) at (0.5,0) {};
	\node[style={shape=circle,fill=black,scale=0.2}] (B1) at (0.5,0.5) {};
	\node[style={shape=circle,fill=black,scale=0.2}] (C2) at (1,0) {};
	\node[style={shape=circle,fill=black,scale=0.2}] (C1) at (1,0.5) {};
	\draw[->] (A1) -- (A2);
	\draw[->] (B2) -- (B1);
	\draw[->] (C1) -- (C2);
	\path[->] (1.3,0.25) edge node[above] {\tiny{$L$}} (1.7,0.25);
	\node[style={shape=circle,fill=black,scale=0.2}] (A2) at (2,0) {};
	\node[style={shape=circle,fill=black,scale=0.2}] (A1) at (2,0.5) {};
	\node[style={shape=circle,fill=black,scale=0.2}] (B2) at (2.5,0) {};
	\node[style={shape=circle,fill=black,scale=0.2}] (B1) at (2.5,0.5) {};
	\node[style={shape=circle,fill=black,scale=0.2}] (C2) at (3,0) {};
	\node[style={shape=circle,fill=black,scale=0.2}] (C1) at (3,0.5) {};
	\draw[->] (A1) -- (A2);
	\draw[->] (B1) -- (C1);
	\draw[->] (C2) -- (B2);
	\node (e) at (3.5,0.25) {~$=LC$};
	\end{tikzpicture}
\end{figure}
\end{columns}
\end{proof}
\end{frame}

\begin{frame}
	\begin{theorem}
		Элементы класса 0+ образуют группу, изоморфную $\mathbb{Z}_6$.	
	\end{theorem}
	\begin{proof}
		\vspace*{-7mm}
		\begin{multline*}
		$$
		M_{corners}=L^{-1}U^{-1}L^{-1}UL^{-1}U^{-1}L^2ULU^{-1}L^2ULU^{-1}LU^2L \\
		(U^{-1}L^{-1})^2L^{-1}ULUL^{-1}U^2L^2U^2LU^{-1}LUL^2U^{-1} \\
		(U^{-1}L^{-1})^2(UL)^2UL^{-1}UL(U^{-1}L^{-1})^2UL^2U^{-1}L^{-1}UL^{-1}
		$$
		\end{multline*}
		\RubikCubeSolved
		\RubikRotation{Lp,Up,Lp,U,Lp,Up,L2,U,L,Up,L2,U,L,Up,L,U2,L,Up,Lp,Up,Lp,Lp,U,L,U,Lp,U2,L2,U2,L,Up,L,U,L2,Up,Up,Lp,Up,Lp,U,L,U,L,U,Lp,U,L,Up,Lp,Up,Lp,U,L2,Up,Lp,U,Lp}
		\RubikRotation{y3,z3}
		\vspace*{-1cm}
		\begin{figure}[c]
		\ShowCube{2cm}{0.3}{\DrawRubikCubeF}
		\caption{Движение $M_{corners}$}
		\end{figure}
		\vspace*{-7mm}
		\end{proof}
\end{frame}
%------------------------------------------------

\begin{frame}
%\vspace*{-0.7cm}	
\begin{multline*} %(1,6,4,3,2)
$$
\alpha=U^{-1}L^{-1}ULU^{-1}L^{-1}UL^{-1}U^{-1}LUL^{-1}U(LU^{-1})^2L^{-1}U^{-1}L^{-1}(UL)^3L^2 \\
U^{-1}L^{-1}UL^{-1}U^{-1}LUL^{-1}U^{-1}L^2UL^{-1}U^{-1}L^2ULU^{-1}LUL
$$
\end{multline*}
\begin{multline*} %(1,2,6,4)
$$
\beta=LU^{-1}LUL^2U^{-1}LUL^2UL^{-1}U^{-1}L(U^{-1}L^{-1})^2U^{-1}(LU)^2L^{-1}U^{-1}L^{-1}\\ 
(L^{-1}ULU^{-1}LU)^2L^2U^{-1}L^{-1}U^{-1}(L^{-1}U)^2LU^2L(UL^{-1})^2U^{-1}
$$
\end{multline*}
\begin{columns}
	\column{0.45\textwidth}
	\RubikCubeSolved
	\RubikRotation{Up,Lp,U,L,Up,Lp,U,Lp,Up,L,U,Lp,U,L,Up,L,Up,Lp,Up,Lp,U,L,U,L,U,L,L2,Up,Lp,U,Lp,Up,L,U,Lp,Up,L2,U,Lp,Up,L2,U,L,Up,L,U,L}
	\RubikRotation{y3,z3}
	\vspace*{-1cm}
	\begin{figure}[c]
		\ShowCube{2cm}{0.3}{\DrawRubikCubeF}
		\caption{Движение $\alpha$}
	\end{figure}
	\column{0.45\textwidth}
	\RubikCubeSolved
	\RubikRotation{L,Up,L,U,L2,Up,L,U,L2,U,Lp,Up,L,Up,Lp,Up,Lp,Up,L,U,L,U,Lp,Up,Lp,Lp,U,L,Up,L,U,Lp,U,L,Up,L,U,L2,Up,Lp,Up,Lp,U,Lp,U,L,U2,L,U,Lp,U,Lp,Up}
	\RubikRotation{y3,z3}
	\vspace*{-1cm}
	\begin{figure}[c]
		\ShowCube{2cm}{0.3}{\DrawRubikCubeF}
		\caption{Движение $\beta$}
	\end{figure}
\end{columns}
%\vspace*{-0.4cm}
\begin{columns}
\column{0.45\textwidth}
\begin{itemize}
\item $\alpha^5=e$
\item $\beta^4=e$
\item $\beta\alpha\beta^{-1}=\alpha^2$
\end{itemize}
\column{0.45\textwidth}
Поэтому в группе содержится группа, изоморфная $F_5$.
\end{columns}
\end{frame}

%------------------------------------------------

\begin{frame}
\begin{theorem}
	Элементы в $F_5$ принадлежат разным подклассам.
\end{theorem}
\begin{proof}
\begin{itemize}
	\item $h$ и $h^{-1}$ не могут быть в разных подклассах
	$$ah^{-1}=h$$
	$$a=h^2$$
	$$a,a^2,\ldots,a^6=e\in G\text{ и }\in F_5\text{, но }\mathbb{Z}_6\not\subset F_5$$
	\item
	Если два элемента $h_1$ и $h_2$ принадлежат одному классу, то $ah_1=h_2$ для некоторого $a\in G_{0+}$ и $a=h_2h_1^{-1}$ $$a^{-1}=h_1h_2^{-1}\in G_{0+}\text{ и }a^{-1}\in F_5$$
\end{itemize}
\vspace*{-0.4cm}
\end{proof}
$\mathbb{Z}_6\cap F_5=\{e\}$, и $G_{corners}=\mathbb{Z}_6F_5=\{hk|h\in\mathbb{Z}_6,k\in F_5\}$.
\end{frame}

%------------------------------------------------

\begin{frame}
\begin{theorem}
$S_5=\mathbb{Z}_6F_5$
\end{theorem}
\begin{proof}
	\begin{itemize}
		\item $F_5$ разбивает $S_5$ на 6 классов смежности.
		%\item $x$ и $x^{-1}$ лежат в разных классах. $$xK=x^{-1}K$$ $$x^2K=K\text{, то есть }x=e$$
		\item Предположим, что $x_1x_2K=x_1K$. $$x_2K=K\text{, поэтому } x_2=e\text{.}$$
		\item Предположим, что $x_1x_2K=x_2K$. $x_2^{-1}x_1x_2\in K$
		$$K_i^{x_2^{-1}}=K_0\text{, то есть } x=e$$
	\end{itemize}
%\vspace*{-0.4cm}
\end{proof}
\end{frame}

%------------------------------------------------

\begin{frame}
\frametitle{Описание полной группы}
\end{frame}

%------------------------------------------------

\begin{frame}
\Huge{\centerline{Спасибо за внимание!}}
\end{frame}

%----------------------------------------------------------------------------------------

\end{document} 
\grid
